\documentclass[dvipdfmx]{jsarticle}

\begin{document}

\title{shannonn}

\section{シャノンの理論}
1948/1949年に発表された情報理論におけるシャノン理論は、以下の3つの部分から構成される。

\begin{enumerate}

  \item サンプリング : 連続する関数をサンプリングすることによって離散正則ベクトルを得る条件を
        検討する。信号を表す離散的な実数値は通常、有限なアルファベットの組み合わせを得るために、
        有限な精度で量子化される。

  \item 情報源符号化 :  記号の組み合わせを2進系列に変換する最適な方法を検討する。
        確率分布に従い、最小化した符号を得ることができる。

  \item 通信路符号化 :  通信中の攻撃やエラーに対する堅牢性を得るために符号に冗長性を加える方法を検討する。
        一般に誤り訂正符号理論と呼ばれる。

\end{enumerate}

\subsection{アナログ信号と離散信号}
数値計算ツールを開発しパフォーマンスを解析するのに、連続集合(アナログ信号と呼ばれる)を介した
数学モデルが用いられる。この連続集合はマイクやデジタルカメラ、MRIなどのハードウェアから得られる
物理世界の信号を表す狙いもある。


\end{document}